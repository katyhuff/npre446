\documentclass[11pt]{article}
\usepackage[inner=1in,outer=1in,top=1in,bottom=1in]{geometry}
\pagestyle{empty}
\usepackage{placeins}
\usepackage{graphicx}
\usepackage{fancyhdr, lastpage, bbding, pmboxdraw}
\usepackage{amsmath, amssymb}
\usepackage[usenames,dvipsnames]{color}
\definecolor{darkblue}{rgb}{0,0,.6}
\definecolor{darkred}{rgb}{.7,0,0}
\definecolor{darkgreen}{rgb}{0,.6,0}
\definecolor{red}{rgb}{.98,0,0}
\usepackage[colorlinks,pdfusetitle,urlcolor=darkblue,citecolor=darkblue,linkcolor=darkred,bookmarksnumbered,plainpages=false]{hyperref}
%\renewcommand{\thefootnote}{\fnsymbol{footnote}}

\pagestyle{fancyplain}
\fancyhf{}
\lhead{ \fancyplain{}{\CourseTitle} }
%\chead{ \fancyplain{}{} }
\rhead{ \fancyplain{}{\CourseSemester \CourseYear} }
%\rfoot{\fancyplain{}{page \thepage\ of \pageref{LastPage}}}
\fancyfoot[RO, LE] {page \thepage\ of \pageref{LastPage} }
\thispagestyle{plain}
\usepackage{tabularx}


%%%%%%%%%%%%%%%%%%%%%%%%%%%%%%%%%%%%
\usepackage{xspace}

\newcommand{\CourseNumber}{NPRE446}
\newcommand{\CourseTitle}{Radiation Interactions With Matter I\xspace}%
\newcommand{\CourseInstructor}{Prof. Kathryn Huff\xspace}%
\newcommand{\CourseSemester}{Fall\xspace}%
\newcommand{\CourseYear}{2019\xspace}%
\newcommand{\CourseDays}{MWF\xspace}%
\newcommand{\CourseStart}{9:00\xspace}%
\newcommand{\CourseEnd}{10:00\xspace}%
\newcommand{\CourseInstructorEmail}{kdhuff@illinois.edu}
\newcommand{\CourseRoom}{305\xspace}%
\newcommand{\CourseBuilding}{Materials Science\xspace}%
\newcommand{\CourseUniversity}{University of Illinois, Urbana-Champaign\xspace}%
\newcommand{\TeachingAssistant}{Natalie Gaughan\xspace}%
\newcommand{\TAOfficeHourDays}{TBD\xspace}%
\newcommand{\TAEmail}{ncg5@illinois.edu}%
\newcommand{\TAOfficeHourStart}{TBD\xspace}%
\newcommand{\TAOfficeHourEnd}{TBD\xspace}%
\newcommand{\TAOfficeHourPlace}{TBD Talbot Laboratory\xspace}
%\newcommand{\HuffOfficeHourDays}{Fridays\xspace}%
%\newcommand{\HuffOfficeHourStart}{3:00pm\xspace}%
%\newcommand{\HuffOfficeHourEnd}{5:00pm\xspace}%
\newcommand{\HuffOfficeHourPlace}{118 Talbot Laboratory, 104 S. Wright St.\xspace}
%\newcommand{\Course<++>}{<++>}
%\newcommand{\Course<++>}{<++>}
%%%%%%%%%%%%%%%%%%%%%%%%%%%%%%%%%%%%
\title{\CourseNumber: \CourseTitle\\}
\author{\CourseUniversity}
\date{\CourseSemester \CourseYear}
\begin{document}
\maketitle
%\setlength{\unitlength}{1in}
\renewcommand{\arraystretch}{1.5}
\begin{center}
\begin{table}[h]
\begin{tabularx}{\textwidth}{rXrX}
\hline
\textbf{Instructor:} & \CourseInstructor & \textbf{Time:} & \CourseDays \CourseStart -- \CourseEnd \\
\textbf{Email:} &  \href{mailto:\CourseInstructorEmail}{\CourseInstructorEmail} & \textbf{Place:} & \CourseRoom \CourseBuilding\\
\textbf{Teaching Assistant} & \TeachingAssistant & & \\
\textbf{Email:} &  \href{mailto:\TAEmail}{\TAEmail} & & \\
\hline
\end{tabularx}
\end{table}
\end{center}

\paragraph{Course Pages:}
\begin{enumerate}
        \item \url{https://compass2g.illinois.edu}
        \item \url{https://github.com/katyhuff/npre446}
        \item \url{https://katyhuff.youcanbook.me}
\end{enumerate}

\paragraph{TA Office Hours:} The teaching assistant for the course, 
\TeachingAssistant, will hold office hours \TAOfficeHourDays from 
\TAOfficeHourStart to \TAOfficeHourEnd in \TAOfficeHourPlace.

\paragraph{Office Hours:} Prof. Huff will hold office hours by appointment 
only, in her office, \HuffOfficeHourPlace. Please make use of the teaching 
assistant and your colleagues before booking an appointment with Prof. Huff. 
You can make  an appointment at \url{katyhuff.youcanbook.me}.

\paragraph{Main References:}
A few essential references for this course will be assigned as readings. The 
required texts for this course are \cite{griffiths_introduction_2004} and 
\cite{yip_nuclear_2014}. The more recent \cite{griffiths_introduction_2018} is not 
required, but can be used instead of the second edition. Additional,a
recommended texts include \cite{krane_introductory_1987,eisberg_quantum_1985,evans_atomic_1955}.
\bibliographystyle{unsrt}
\renewcommand{\refname}{\normalfont\selectfont\normalsize}\vspace{-1cm} 
\bibliography{bibliography}

\paragraph{Description:}
The classical and quantum theories of the interaction of radiation (neutrons, photons, and
charged particles) with matter are the core components of nuclear and materials science
and engineering. At UIUC, we offer a sequence of four courses (446, 447, 521, and 529)
at progressively deepening levels on this subject. The sequence, in the aggregate, aims to
provide the students with solid training in essential physical principles, mathematical
competence, and computational skills. In this course, we provide a quantitative
introduction to introductory quantum mechanics, fundamentals of atomic and nuclear
physics, and interactions of radiation with matter.

\paragraph{Objectives:} 

\paragraph{The quantum physics elements of this course will equip students to}

\begin{itemize}
        \item Recognize the limitations of classical theory  (stable atomic 
                model, black-body radiation, the photoelectric effect, and 
                Compton scattering).
        \item Apply classical theory to fundamental problems.
        \item Explain the implications of wave-particle duality.
        \item Apply operators in physical equations.
        \item Solve the Schr\"odinger equation for canonical problems.
        \item Understand and apply fundamental concepts (eigenstates, 
                observables, statistical interpretation, probability 
                conservation, bound/unbound states).
        \item Identify the bound and unbound states in key potentials: square
potential, Harmonic oscillator, ladder operators, free particle, $\delta$ 
                potential.
        \item Define, derive, and apply the uncertainty principle.
\end{itemize}

\paragraph{The atomic and nuclear physics elements of this course will equip students to:}
\begin{itemize}
        \item Fundamentally describe atomic structure and nuclear properties. 
        \item Understand and use nuclear physics terminology.
        \item Differentiate nuclei and interactions via their size, 
                distributions of charge and mass, abundance, and binding energy.
        \item Identify and calculate nuclear phenomena via Q-value, separation energy, semi-empirical
mass formula, liquid drop model, mass parabola, spin, parity, electromagnetic moments), nuclear force and
nuclear structure.
                \item Define and apply the properties of nuclear forces, deuteron structure, neutron-proton scattering, phase shift,
partial wave approximation, scattering length, differential cross section, exchange force model, nuclear
shell model, nuclear magic numbers.
                \item Model and solve equations related to radioactive decay 
                        and binary nuclear reactions.
\end{itemize}

\paragraph{The interaction of radiation with matter portion of this course will equip 
students to:} 
\begin{itemize}
        \item Categorize and characterize neutron interations with matter. 
        \item Identify and quantify neutron-proton scattering, energy 
                dependence of cross sections, and moderation.
        \item Categorize and characterize gamma interactions with matter.
        \item Identify and quantify gamma attenuation, the photoelectric 
                effect, Compton scattering, pair production.
        \item Categorize and characterize heavy and light charged particle 
                interactions with matter.
        \item Identify and calculate the features of charged particle stopping 
                power, Bragg curve, range, ionization loss, and radiation loss.
\end{itemize}

\paragraph{Prerequisites:} 
\begin{itemize}
        \item Junior standing is recommended.
        \item MATH 285
        \item PHYS 211 - 214, or equivalent.
        \item Linear Algebra (MATH 125) is not required, but highly recommended.
\end{itemize}

\paragraph{Grading Policy:} Grades will be assigned as a weighted sum of the following work. 

\begin{table}[h]
\begin{tabularx}{\textwidth}{Xr}
\textbf{Work} & \textbf{Weight}\\
\hline
\textbf{Homework} & (35\%) \\
\textbf{Quizzes} & (5\%) \\
\textbf{Midterm 1} & (15\%) \\
\textbf{Midterm 2} & (15\%) \\
\textbf{Final} & (30\%) \\
\hline
\textbf{Total} & (100\%) \\
\end{tabularx}
\end{table}

\paragraph{Important Dates:}

The following dates are subject to change.
\begin{center} \begin{minipage}{3.8in}
\begin{flushleft}
Midterm \#1      \dotfill September 25, 2019, 9:00am-10:00am\\
Midterm \#2      \dotfill November 6, 2019, 9:00am-10:00am  \\
Final Exam      \dotfill December 18, 2019, 1:30pm-4:00pm\\
\end{flushleft}
\end{minipage}
\end{center}

\paragraph{Class Policies:}  

\begin{itemize}
\item[] \textbf{Integrity:} This is an institution of higher
learning. You will be swiftly ejected from the course if you are caught
undermining its integrity. Note the
\href{http://www.provost.illinois.edu/academicintegrity/students.html}{Student's
Quick Reference Guide to Academic Integrity} and the
\href{http://studentcode.illinois.edu/article1_part4_1-401.html}{Academic
Integrity Policy and Procedure}.  
\item[] \textbf{Attendance:} Regular attendance is mandatory. Request approval 
        for absence for extenuating circumstances prior to absence.
\item[] \textbf{Electronics:} Active participation is essential and expected. 
        Accordingly, students must turn off all electronic devices (laptop, 
        tablets, cellphones, etc.) during class. Exceptions may be granted for 
        laptops if engaging in computational exercises or taking notes. 
\item[] \textbf{Collaboration:} Collaboratively reviewing course materials and 
        studying for exams with fellow students can be enriching.  This is 
                recommended.  However, unless otherwise instructed, homework 
                assignments are to be completed independently and materials 
                submitted as homework should be the result of one's own 
                independent work.
\item[] \textbf{Late Work:} Late work has a halflife of 1 hour. That is, 
        adjusted for lateness, your grade $G(t)$ is a decaying percentage of 
                the raw grade $G_0$. An assignment turned in $t$ hours late 
                will receive a grade according to the following relation:
\begin{align*}
        G(t) &= G_0e^{-\lambda t}
        \intertext{where}
        G(t) &= \mbox{grade adjusted for lateness}\\
        G_0 &= \mbox{raw grade}\\
        \lambda &= \frac{ln(2)}{t_{\frac{1}{2}}} = \mbox{decay constant} \\
        t &= \mbox{time elapsed since due [hours]}\\
        t_{1/2} &= 1 = \mbox{half-life [hours]} \\
\end{align*}
\item[] \textbf{Make-up Work:} There will be no negotiation about late work 
        except in the case of absence documented by an absence letter from the 
                Dean of Students.  The university policy for requesting such a 
                letter is in 
                \href{http://studentcode.illinois.edu/article1_part5_1-501.html}{the 
                Student Code}. Please note that such a letter is appropriate 
                for many types of conflicts, but that religious conflicts 
                require special early handling. In accordance with university 
                policy, students seeking an excused absence for religious 
                reasons should complete the Request for Accommodation for 
                Religious Observances Form, which can be found on the Office of 
                the Dean of Students website. The student should submit this 
                form to the instructor and the Office of the Dean of Students 
                by the end of the second week of the course to which it 
                applies. 

\item[] \textbf{Grade Disputes:} It is important that you understand and agree
        with the grade you receive on assignments and exams. If you would like
        to dispute your score, you must send an explanation by email to Prof.
        Huff within one week of recieving the grade.
        \textbf{Do not expect me to regrade anything while in conversation with
        you} as that would not be fair to the other students in the class, whose
        homeworks were graded without them present.  If you request a regrade,
        be aware that the entire assignment will be regraded and is subject to
        double-jeopardy: it is possible that your score will go down.
        Regrade requests should be based on an error on my part (e.g., adding
        up the points incorrectly) or what you suspect is a misunderstanding of
        your work (e.g., arriving at the correct answer using an unexpected
        technique). Regrade requests that argue with the rubric (e.g., ``this is
        wrong, but you took too many points off'') will be returned without
        consideration.
        \textbf{Your work should stand alone.} If an assignment is disorganized or
        ambiguous, and requires an extensive explanation to the grader, you
        will likely still lose points. The homeworks not only evaluate your
        understanding of the material - they also evaluate your ability to
        communicate that understanding clearly and concisely.

\end{itemize}

\paragraph{Accessibility:} I hope that this course will be inclusive and 
accommodating for all learners. As such, I am committed upholding the vision 
and values of \href{http://www.inclusiveillinois.illinois.edu/index.html}{Inclusive Illinois}
in my 
classroom.  With regard to accommodating all learners, please note that many 
resources are provided through 
\href{http://disability.illinois.edu/academic-support/accommodations}{the 
Division of Disability Resources and Educational Services}.  To request 
particular accommodations, please contact me as soon as possible so that we can 
work out any necessary arrangements.

\paragraph{Other Resources:} 
University students typically experience a wide range of stressors during their 
time on campus. Accordingly, campus resources exist to help students manage  
stress levels, mental health, physical health, and emergencies while navigating 
this environment. I hope you will take advantage of these campus resources as 
soon as they can be of help.

\begin{itemize}
\item \href{https://campusrec.illinois.edu/}{The Campus Recreational Centers}
\item \href{http://counselingcenter.illinois.edu/}{The Counselling Center}
\item \href{https://mckinley.illinois.edu/}{The McKinley Health Clinic}
\item \href{http://www.mckinley.illinois.edu/medical-services/mental-health}{The McKinley Mental Health Clinic}
\item \href{https://odos.illinois.edu/community-of-care/emergency-dean/}{The Emergency Dean}
\end{itemize}


\paragraph{Run. Hide. Fight.}
It is important that we take time to prepare for a situation in which our 
safety could depend on our ability to react quickly. Please review the 
university guidance on responding to emergency situations 
\url{https://police.illinois.edu/emergency-preparedness/run-hide-fight/}.
Take a moment to learn the different ways to leave
this building. If there’s ever a fire alarm or something like that, you’ll know
how to get out and you’ll be able to help others get out. Next, figure out the
best place to go in case of severe weather - we’ll need to go to a low-level in
the middle of the building, away from windows. And finally, if there’s ever
someone trying to hurt us, our best option is to run out of the building. If we
cannot do that safely, we’ll want to hide somewhere we can’t be seen, and we’ll
have to lock or barricade the door if possible and be as quiet as we can. We
will not leave that safe area until we get an Illini-Alert confirming that it’s
safe to do so. If we can’t run or hide, we’ll fight back with whatever we can
get our hands on. If you want to better prepare yourself for any of these
situations, visit \url{police.illinois.edu/safe}. Remember you can sign up for
emergency text messages at \url{emergency.illinois.edu}.



\pagebreak
\FloatBarrier
\renewcommand{\arraystretch}{1}
\begin{table}[h]
\begin{center}
\begin{tabular}{lllcllll}
\multicolumn{8}{c}{\textbf{Course Schedule:}\textit{ This schedule is subject to change}}\\
\hline
\textbf{Date} & \textbf{Week} & \textbf{Day} & \textbf{Unit} & \textbf{Chap.} & \textbf{Quiz} & \textbf{HW} & \textbf{HW}\\
 &  &  &  &  &                                                                  \textbf{Due}  & \textbf{Given} & \textbf{Due}\\
\hline
\hline
08-26 & 1 & M & Intro & G1 &  &  & \\
08-28 & 1 & W & Wave Function & G1 &  &  & \\
08-30 & 1 & F & Wave Function & G1 &  & HW1 & \\
09-02 & 2 & M & \textbullet~\textbf{No Class} \textbullet & & Q1 &  & \\
09-04 & 2 & W & Wave Function & G1 &  &     &    \\
09-06 & 2 & F & Wave Function & G1 &  &     &    \\
09-09 & 3 & M & Wave Function & G1 & Q2 &  & \\
09-11 & 3 & W & Time-Independent Schr\"odinger & G2 &  &  & \\
09-13 & 3 & F & Time-Independent Schr\"odinger  & G2 &  & HW2 & HW1\\
09-16 & 4 & M & Time-Independent Schr\"odinger  & G2 & Q3 &  & \\
09-18 & 4 & W & Time-Independent Schr\"odinger & G2 &  &  & \\
09-20 & 4 & F & Time-Independent Schr\"odinger  & G2 &  & HW3 & HW2 \\
09-23 & 5 & M & Review &  & Q4 &  & \\
09-25 & 5 & W & \textbullet~\textbf{Midterm} \textbullet  & G1, G2  &  &  & \\
09-27 & 5 & F & Formalism & G3 &  &  & \\
09-30 & 6 & M & Nuclear Properties & Y3 & Q5 &  & \\
10-02 & 6 & W & Nuclear Properties & Y3 &  &  & \\
10-04 & 6 & F & Nuclear Properties& Y4 &  & HW4 & HW3\\
10-07 & 7 & M & Nuclear Properties & Y4 & Q6 &  & \\
10-09 & 7 & W & Stability of Nuclei & Y4 &  &  & \\
10-11 & 7 & F & Stability of Nuclei & Y5 &  & HW5 & HW4\\
10-14 & 8 & M & Energy Level Models & Y5 & Q7 &  & \\
10-16 & 8 & W & Energy Level Models & Y5 &  &  & \\
10-18 & 8 & F & Nuclear Decay & Y6 &  &  & \\
10-21 & 9 & M & Collision Cross Sections & Y7 & Q8 &  & \\
10-23 & 9 & W & Collision Cross Sections & Y7 &  &  & \\
10-25 & 9 & F & Collision Cross Sections & Y7 &  & HW6 & HW5\\
10-28 & 10 & M & Collision Cross Sections & Y7 & Q9 &  & \\
10-30 & 10 & W & Collision Cross Sections & Y7 &  &  & \\
11-01 & 10 & F & Neutron Scattering & Y9 & & HW7 & HW6\\
11-04 & 11 & M & Neutron Scattering & Y9 & Q10 &  & \\
11-06 & 11 & W & Review &  & &  & \\
11-08 & 11 & F & \textbullet~\textbf{Midterm} \textbullet & G3, Y4-Y7 & & & \\
11-11 & 12 & M & Neutron Scattering & Y9 & Q11 &  & \\
11-13 & 12 & W & Neutron Scattering  & Y9 & &  & \\
11-15 & 12 & F & $\gamma$ Scattering \& Absorption & Y10 & & HW8 & HW7\\
11-18 & 13 & M & $\gamma$ Scattering \& Absorption & Y10 & Q12 &  & \\
11-20 & 13 & W & $\gamma$ Scattering \& Absorption & Y10 &  &  & \\
11-22 & 13 & F & $\gamma$ Scattering \& Absorption & Y10 &  &  & \\
11-25 & 14 & M & \textbullet~\textbf{No Class} \textbullet &  & Q13 &  & \\
11-27 & 14 & W & \textbullet~\textbf{No Class} \textbullet  &   &    &  & \\
11-29 & 14 & F & \textbullet~\textbf{No Class} \textbullet &  &  & HW9 & HW8\\
12-02 & 15 & M & Charged Particle Stopping & Y11 & Q14 &  & \\
12-04 & 15 & W & Charged Particle Stopping & Y11 &  &  & \\
12-06 & 15 & F & Charged Particle Stopping & Y11 &  &  & HW9\\
12-09 & 16 & M & Charged Particle Stopping & Y11 &  &  & \\
12-11 & 16 & W & Review &  &  &  & \\
12-18 & 17 & W & \textbullet~\textbf{Final Exam} \textbullet &  &  &  & \\
\end{tabular}
\end{center}
\end{table}
\FloatBarrier


%%%%%% THE END 
\end{document} 
